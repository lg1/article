\section{Materials and methods}
\label{sec:methods}

\subsection{Preparartion of solid substrate for patterning of cardiac cells monolayer}
\subsubsection{Materials}
	\label{sec:specimen_preparation_electrospinning}


PCL(polycaprolactone, Sigma 65,000 MW) was dissolved in hexafluorisopropanol at 5\%concentration.  
Polydimethylsiloxane (PDMS) was prepared by the thorough mixing of two liquid components (purchased from Dow Corning Toray Corp., Japan) at a ratio of 1:10 in a flat Petri dish. After being kept at 80\gc\space for 1 hour, the polymerized PDMS layer was separated from the dish, cut into 15x15 mm squares.

All chemicals were purchased from Wako Chemicals Inc.,Japan.
\subsubsection{Electrospinning of PCL fibers}
The prepared solution of PCL polymer was electrospun using Nanon electrospinning setup(MECC CO.,LTD) with 8kV between needle and grounded collector. The solution was ejected through 20 gauge blut-tip needle at feed rate of 1 ml/h. Fibers were deposited directly during the process of electrospining on either PDMS sheets or 22mm diameter cover glass pieces(Corning Corp.), which were placed on the surface of grounded stationary collector, representing a huge metal plate.    
\subsubsection{Creating of randomly orientated nanofibrous substrates }
In order to obtain samples of different nanofibers positioning density we varied spraying time in the range of 10 to 60 seconds. Thus we produced 2-D fibrous meshes with pore size of 5-500 um. 
